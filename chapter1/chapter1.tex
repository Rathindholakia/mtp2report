\chapter{Introduction}
\pagenumbering{arabic}  % 1, 2, 3, 4, ...
\setcounter{page}{1}

Electric energy has become most important source of energy and is widely used resource in present time, with ever increasing demand of the resource it becomes more and more difficult to maintain the system and Power System is no exception. Power System has become an complex entity and has gone beyond the limit of manual operattion and control which makes automation and "smart" control imparitive. This creates demand for new set of measurement, operation and control tools. Out of this tools measurement tools are the most fundaental building block of the modern power system which is also know populalry as "smart grid". They are the "eyes" and "ears" in the system to the centralized operating-control-corrective brain system.  

In power system active power and frequency are the most important parameters to be monitored, flow of active power is decided by the phase angle of voltage between buses. Flow of active power decides the structure of network (transmission lines, capacity of devices etc) and hence accurate measurement of it has been of great interest since 1980s.\cite{agphadkebook}. Conventionally relative phase angle between buses in the network, due to limitation of telecommunication links, computational power and the economic pheasibility. This method(s) were slow, fairly accurate and dependent on a tones of heavy manual calcualtion. 
After advancements in communition channels and their speed \& reliability, better computation and satelite availability, trend of absolute phase difference measurement came in to existance.   
\section{Phasors, Synchrophasors and PMUs}
\subsection{Phasors: Defination}
In Physics and engineering, \textit{phasor} is a complex number representing a sinusoidal quantity whose amplitude (A), angular velocity ($\omega$) and initial phase ($\phi$) are time-invarient.It is an analytic representation which decomposes sine function in to product of complex constants and a factor which encapsulates the frequency and time dependence.
\section{Chapter 1 Section 2}
